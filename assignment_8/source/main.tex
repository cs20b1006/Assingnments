\documentclass{article}
\usepackage[utf8]{inputenc}
\usepackage{karnaugh-map}

% ADD TITLE HERE
\title{K-map for CBSE 2019 6.D}
\author{Sharon Kachhi }
%\date{}
\begin{document}

\maketitle

\section{Question}
Reduce the following Boolean expression to its simplest form using K-map:
\begin{equation}
  F(P,Q,R,S)\;=\;\sum(0,1,2,3,5,6,7,10,14,15)  
\end{equation}
\section{Boolean equation - SOP form}
\begin{equation}
F\;=\;\overline{P}Q\;+\;R\overline{S}\;+\;\overline{P}S\;+\;QR
\end{equation}
 
\section{K-Map} 
\begin{figure}[h]
    \centering
\begin{karnaugh-map}[4][4][1][][]
    \maxterms{4,8,9,11,12,13}
    \minterms{0,1,2,3,5,6,7,10,14,15}
    \implicant{4}{12}
    \implicant{12}{9}
    \implicant{9}{11}
    \draw[color=black, ultra thin] (0, 4) --
    node [pos=0.7, above right, anchor=south west] {$RS$} 
    node [pos=0.7, below left, anchor=north east] {$PQ$} 
    ++(135:1);
\end{karnaugh-map}
    \caption{K-map for eq.(2)}
    \label{fig:my_label}
\end{figure}

\section{Boolean Equation - POS form}
We can derive the following POS from the above K-map
\begin{equation}
    F\;=\;(\overline{P}\;+\;R)(\overline{Q}\;+R\;+\;S)(\overline{P}\;+\;Q\;+\;\overline{S})
\end{equation}

\end{document}