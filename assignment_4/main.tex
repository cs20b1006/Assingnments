\documentclass{article}
\usepackage{xcolor}
\usepackage{listings}

\title{Assignment 4}
\author{Sharon Kachhi}

\definecolor{mGreen}{rgb}{0,0.6,0}
\definecolor{mGray}{rgb}{0.5,0.5,0.5}
\definecolor{mPurple}{rgb}{0.58,0,0.82}
\definecolor{backgroundColour}{rgb}{0.95,0.95,0.92}

\lstdefinestyle{CStyle}{
    backgroundcolor=\color{backgroundColour},   
    commentstyle=\color{mGreen},
    keywordstyle=\color{magenta},
    numberstyle=\tiny\color{mGray},
    stringstyle=\color{mPurple},
    basicstyle=\footnotesize,
    breakatwhitespace=false,         
    breaklines=true,                 
    captionpos=b,                    
    keepspaces=true,                 
    numbers=left,                    
    numbersep=5pt,                  
    showspaces=false,                
    showstringspaces=false,
    showtabs=false,                  
    tabsize=2,
    language=C
}



\begin{document}
\maketitle
\section{Equation for D:}
\begin{equation}
    D = W.X.Y.\overline{Z}+\overline{W}.\overline{X}.\overline{Y}.Z
\end{equation}
\section{C code:}
\begin{lstlisting}[style=CStyle]
//Assignment 4 
//submitted by Sharon Kachhi

#include <stdio.h>

//The  main function
int main(void)
{

//2 bits = 1 baud
//4 bits = 1 nibble
//8 bits = 1 byte

//unsigned char takes input as 1 byte

unsigned char  Z=0x00,Y=0x01,X=0x01,W=0x01;//inputs in hex	
unsigned char one = 0x01;//used for displaying the output in bit
unsigned char A,B,C,D;//outputs

A = ((~W)&(~X)&(~Y)&(~Z))|((~W)&(X)&(~Y)&(~Z))|((~W)&(~X)&Y&(~Z))|((~W)&X&Y&(~Z))|((~W)&(~X)&(~Y)&(Z));
//Boolean function for A
B = ((W)&(~X)&(~Y)&(~Z))|((~W)&(X)&(~Y)&(~Z))|((W)&(~X)&(Y)&(~Z))|((~W)&(X)&(Y)&(~Z));
C = ((W)&(X)&(~Y)&(~Z))|((~W)&(~X)&(Y)&(~Z))|((W)&(~X)&(Y)&(~Z))|((~W)&(X)&(Y)&(~Z));
D = (W&X&Y&(~Z))|((~W)&(~X)&(~Y)&Z);//Boolean function for D

printf("%x\n",one&A);//Output A
printf("%x\n",one&B);//Output B
printf("%x\n",one&C);//Output C
printf("%x\n",one&D);//Output D
return 0;
}
\end{lstlisting}
\end{document} 